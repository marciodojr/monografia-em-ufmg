% !TEX root = Monografia.tex

\chapter*[Introdução]{Introdução}
\addcontentsline{toc}{chapter}{Introdução}
\label{chap:introducao}

	O modo como o ser humano interage com o ambiente é fortemente associado com suas experiências passadas. Quando uma criança vê uma bolinha sendo arremessada verticalmente, ela pode achar o movimento de subida e descida engraçado ou curioso. Instintivamente, ao repetir o experimento, a criança esperará o mesmo resultado e ficará intrigada caso não o obtenha. Situação similar ocorre em truques de mágica, onde o mágico tira um coelho de sua cartola, surpreendendo adultos e crianças.

	Com conhecimento das leis de Newton a primeira situação deixa de algo inicialmente inesperado e passa a ser algo previsível. A segunda situação, quando se descobre o segredo do truque usado pelo mágico, deixa de ser ``mágica'' e passa a ser uma encenação para direcionar a atenção do público de modo que não perceba o que realmente está acontecendo. Ainda sim, podemos dizer que o truque de mágica é um feito surpreendente, pois enganar uma plateia enorme e atenta não é fácil, porém, já não se trata de algo mágico.

	A mecânica quântica pode ser vista como a descrição da matéria e da luz em todo o seu detalhe em escala atômica. O comportamento em escala atômica difere de tudo que o ser humano tem experiência direta. Tal comportamento não é puramente ondulatório, não é um comportamento de partícula, não são como nuvens, ou bolas de bilhar, ou pesos em molas, ou qualquer coisa da experiência humana do dia a dia.

	Historicamente, imaginava-se que o elétron se comportava como partícula, mais tarde, observou-se que também se comporta como onda. Como este comportamento dual, pode-se dizer que, no fim, o elétron não se comporta nem como partícula, nem como onda.

	A descrição do comportamento em escala atômica foi feita com precisão por volta dos anos de 1926 e 1927 com os trabalhos de Schrödinger, Heisenberg e Bohr. Schrödinger introduziu a formulação ondulatória através de uma equação que descreve um sistema quântico em determinado tempo. Heisenberg introduziu o princípio da incerteza, apresentando limites para o que pode ser medido em um sistema quântico. Born apresentou a interpretação probabilística da função de onda. Em conjunto com contribuições de Dirac, Bohr e Jordan tais ideias culminaram na interpretação de Copenhague da mecânica quântica firmada entre 1927 e 1930 \cite{faye2022copenhagen}.

	Por volta do ano de 1936 Alan Turing e Alonzo Church mostraram a noção de computabilidade universal \cite{turing1936computable} e \cite{church1936unsolvable}. Turing apresenta a noção de computabilidade através de determinadas máquinas (que podemos visualizar como algoritmos), mais ainda, demonstra que existe uma máquina universal capaz de executar qualquer par de especificação de máquina e entrada. Assim, a noção de computabilidade é completamente capturada por tais máquinas.

	A partir de 1970 diversas técnicas para controlar sistemas quânticos foram inventadas, por exemplo, o isolamento de um único átomo de modo a permitir a investigação de seu comportamento com grande precisão. Com a possibilidade de controle de tais sistemas e a combinação em sistemas maiores (com grande dificuldade, mas possível), surge o questionamento sobre a utilização das propriedades quânticas para execução de determinadas operações, em outras palavras, para computação.

	No mesmo período a teoria da complexidade computacional é desenvolvida e uma noção formal para solução eficiente e ineficiente é construída e apresentada por Cook \cite{Cook1971} e Levin \cite{Levin1973}.

	O objetivo principal deste trabalho é estudar a base matemática utilizada em algoritmos quânticos e utilizá-la para implementar alguns algoritmos bem conhecidos. Para isso, no capítulo 1 introduzimos conceitos iniciais de álgebra linear. No capítulo 2 os postulados da mecânica quântica e noções de medida. O capítulo 3 é destinado a introdução de circuitos quânticos e à biblioteca de circuitos quânticos da IBM chamada Qiskit. No capítulo 4 são apresentados dois algoritmos quânticos o algoritmo de Grover e o Algoritmo Quântico de Otimização Aproximada (QAOA). O capítulo 5 apresenta as conclusões desta monografia.
