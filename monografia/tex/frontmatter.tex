\selectlanguage{brazil}

% Retira espaço extra obsoleto entre as frases.
\frenchspacing 

% ----------------------------------------------------------
% ELEMENTOS PRÉ-TEXTUAIS
% ----------------------------------------------------------
% \pretextual

% ---
% Capa
% ---
\imprimircapa
% ---

% ---
% Folha de rosto
% (o * indica que haverá a ficha bibliográfica)
% ---
\imprimirfolhaderosto*
% ---

% ---
% Inserir a ficha bibliografica
% ---

% Isto é um exemplo de Ficha Catalográfica, ou ``Dados internacionais de
% catalogação-na-publicação''. Você pode utilizar este modelo como referência. 
% Porém, provavelmente a biblioteca da sua universidade lhe fornecerá um PDF
% com a ficha catalográfica definitiva após a defesa do trabalho. Quando estiver
% com o documento, salve-o como PDF no diretório do seu projeto e substitua todo
% o conteúdo de implementação deste arquivo pelo comando abaixo:
%
% \begin{fichacatalografica}
%     \includepdf{fig_ficha_catalografica.pdf}
% \end{fichacatalografica}

% TODO
% \begin{fichacatalografica}
% 	\sffamily
% 	\vspace*{\fill}					% Posição vertical
% 	\begin{center}					% Minipage Centralizado
% 	\fbox{\begin{minipage}[c][8cm]{13.5cm}		% Largura
% 	\small
% 	\imprimirautor
% 	%Sobrenome, Nome do autor
	
% 	\hspace{0.5cm} \imprimirtitulo  / \imprimirautor. --
% 	\imprimirlocal, \imprimirdata-
	
% 	\hspace{0.5cm} \thelastpage p. : il. (algumas color.) ; 30 cm.\\
	
% 	\hspace{0.5cm} \imprimirorientadorRotulo~\imprimirorientador\\
	
% 	\hspace{0.5cm}
% 	\parbox[t]{\textwidth}{\imprimirtipotrabalho~--~\imprimirinstituicao,
% 	\imprimirdata.}\\
	
% 	\hspace{0.5cm}
% 		1. Palavra-chave1.
% 		2. Palavra-chave2.
% 		2. Palavra-chave3.
% 		I. Orientador.
% 		II. Universidade xxx.
% 		III. Faculdade de xxx.
% 		IV. Título 			
% 	\end{minipage}}
% 	\end{center}
% \end{fichacatalografica}
% ---


% ---
% Inserir folha de aprovação
% ---

% Isto é um exemplo de Folha de aprovação, elemento obrigatório da NBR
% 14724/2011 (seção 4.2.1.3). Você pode utilizar este modelo até a aprovação
% do trabalho. Após isso, substitua todo o conteúdo deste arquivo por uma
% imagem da página assinada pela banca com o comando abaixo:
%
% \begin{folhadeaprovacao}
% \includepdf{folhadeaprovacao_final.pdf}
% \end{folhadeaprovacao}
%

% TODO
% \begin{folhadeaprovacao}

%   \begin{center}
%     {\ABNTEXchapterfont\large\imprimirautor}

%     \vspace*{\fill}\vspace*{\fill}
%     \begin{center}
%       \ABNTEXchapterfont\bfseries\Large\imprimirtitulo
%     \end{center}
%     \vspace*{\fill}
    
%     \hspace{.45\textwidth}
%     \begin{minipage}{.5\textwidth}
%         \imprimirpreambulo
%     \end{minipage}%
%     \vspace*{\fill}
%    \end{center}
        
%    Trabalho aprovado. \imprimirlocal, 24 de novembro de 2012:

%    \assinatura{\textbf{\imprimirorientador} \\ Orientador} 
%    \assinatura{\textbf{Professor} \\ Convidado 1}
%    \assinatura{\textbf{Professor} \\ Convidado 2}
%    %\assinatura{\textbf{Professor} \\ Convidado 3}
%    %\assinatura{\textbf{Professor} \\ Convidado 4}
      
%    \begin{center}
%     \vspace*{0.5cm}
%     {\large\imprimirlocal}
%     \par
%     {\large\imprimirdata}
%     \vspace*{1cm}
%   \end{center}
  
% \end{folhadeaprovacao}
% ---

% ---
% Dedicatória
% ---
% TODO
% \begin{dedicatoria}
%    \vspace*{\fill}
%    \centering
%    \noindent
%    \textit{ Este trabalho é dedicado às crianças adultas que,\\
%    quando pequenas, sonharam em se tornar cientistas.} \vspace*{\fill}
% \end{dedicatoria}
% ---

% ---
% Agradecimentos
% ---
% TODO
\begin{agradecimentos}

Agradeço, primeiramente, ao professor orientador Csaba Schneider, por ter aceitado
o convite para orientar esta monografia, bem como pela dedicação, paciência e
genuíno interesse no tema proposto. Agradeço pela postura descontraída e ao mesmo 
tempo rigorosa em nossas reuniões.

Registro também meus sinceros agradecimentos aos professores das disciplinas
ofertadas no curso de Especialização em Matemática da Universidade Federal de
Minas Gerais. O empenho em lecionar e a paixão pelos temas apresentados
contribuíram de forma decisiva para a minha formação. Em cada disciplina, além
do conteúdo abordado, tive a oportunidade de conhecer diferentes formas de
trabalhar e, certamente, distintas maneiras de compreender e encarar a vida.

Agradeço, ainda, aos responsáveis administrativos do curso, pelo suporte,
atenção e por todo o trabalho que, muitas vezes, permanece invisível, mas cuja
ausência impactaria de forma significativa o bom andamento do curso.

Por fim, deixo um agradecimento especial à minha companheira, Elizandra Cruz,
por me incentivar a retomar os estudos e pela paciência ao longo desse processo,
seja pela distância, seja por me ouvir falar sobre matemática sempre que retornava.

\end{agradecimentos}
% ---


\chapter*{Epígrafe}


\thispagestyle{empty}

\vspace{2cm} % ajuste fino aqui

% ---
% Epígrafe
% ---
% TODO
\begin{epigrafe}
	\begin{flushright}
		\begin{minipage}{0.6\textwidth}
		\itshape
		``We can only see a short distance ahead, but we can see plenty there that needs to be done.''

		\vspace{0.5em}
		\raggedleft
		— Alan Turing
		\end{minipage}
	\end{flushright}
\end{epigrafe}

% ---

% ---
% RESUMOS
% ---

% resumo em português
% TODO
\setlength{\absparsep}{18pt} % ajusta o espaçamento dos parágrafos do resumo
\begin{resumo}

Este trabalho apresenta um estudo introdutório da computação quântica a partir dos conceitos matemáticos de álgebra linear até a a implementação prática de algoritmos quânticos por meio de circuitos utilizando a biblioteca de circuitos quânticos da IBM denominada Qiskit. Inicialmente, são introduzidos conceitos essenciais de álgebra linear, tais como espaços de Hilbert, operadores lineares e produto tensorial, estabelecendo a base matemática mínima necessária para a descrição de sistemas quânticos. Em seguida, são discutidos os postulados da mecânica quântica, com destaque para a evolução unitária, o processo de medição e a descrição de sistemas compostos, fundamentais para a compreensão do comportamento quântico e de fenômenos como superposição e emaranhamento.

Na sequência, o trabalho aborda o modelo de computação baseado em circuitos quânticos, fazendo um paralelo com circuitos clássicos e enfatizando as restrições impostas pela reversibilidade e pela impossibilidade de clonagem de estados quânticos. São apresentados exemplos centrais da computação quântica, como a geração de estados de Bell e o protocolo de teleporte quântico, além de uma introdução prática ao framework Qiskit, utilizado para construção, simulação e execução de circuitos quânticos.

Por fim, são estudados e implementados dois algoritmos quânticos: o algoritmo de Grover, para busca não estruturada, e o Algoritmo Quântico de Otimização Aproximada (QAOA), aplicado ao problema do corte máximo em grafos. Para ambos, são discutidos os princípios teóricos e a correspondência entre a formulação matemática e sua implementação computacional. No caso do QAOA, é apresentada ainda uma justificativa (informal) baseada na aproximação da evolução adiabática e na fórmula de Lie--Trotter. O trabalho busca, aproximar os fundamentos matemáticos com a noção de circuitos clássica e a prática de implementação dos algoritmos quânticos utilizando o Qiskit.

 \textbf{Palavras-chave}: computação quântica; circuitos quânticos; algoritmo de Grover; QAOA; Qiskit.
\end{resumo}

% resumo em inglês
% TODO:
% \begin{resumo}[Abstract]
%  \begin{otherlanguage*}{english}
%    This is the english abstract.

%    \vspace{\onelineskip}
 
%    \noindent 
%    \textbf{Keywords}: latex. abntex. text editoration.
%  \end{otherlanguage*}
% \end{resumo}

% ---
% inserir lista de ilustrações
% ---
\pdfbookmark[0]{\listfigurename}{lof}
\listoffigures*
\cleardoublepage
% ---

% ---
% inserir lista de quadros
% ---
% \pdfbookmark[0]{\listofquadrosname}{loq}
% \listofquadros*
% \cleardoublepage
% ---

% ---
% inserir lista de tabelas
% ---
% \pdfbookmark[0]{\listtablename}{lot}
% \listoftables*
% \cleardoublepage
% ---

% ---
% inserir lista de abreviaturas e siglas
% ---
% TODO
% \begin{siglas}
%   \item[ABNT] Associação Brasileira de Normas Técnicas
%   \item[abnTeX] ABsurdas Normas para TeX
% \end{siglas}
% ---

% ---
% inserir lista de símbolos
% ---
% TODO
\begin{simbolos}

  \item[$\mathbb{C}$] Conjunto dos números complexos
  \item[$\mathbb{R}$] Conjunto dos números reais

  \item[$\mathcal{H}$] Espaço de Hilbert complexo associado a um sistema quântico
  \item[$\dim(\mathcal{H})$] Dimensão do espaço de Hilbert

  \item[$\ket{\psi}$] Vetor de estado quântico (ket)
  \item[$\bra{\psi}$] Vetor dual associado a $\ket{\psi}$ (bra)
  \item[$\braket{\phi|\psi}$] Produto interno entre os estados $\ket{\phi}$ e $\ket{\psi}$
  \item[$\ket{\phi}\bra{\psi}$] Produto externo (ou produto ``ket por bra'')

  \item[$\otimes$] Produto tensorial
  \item[$I$] Operador identidade
  \item[$U$] Operador unitário

  \item[$H$] Porta de Hadamard
  \item[$X$] Matriz de Pauli $X$
  \item[$Y$] Matriz de Pauli $Y$
  \item[$Z$] Matriz de Pauli $Z$
  \item[$CX$] Porta controlada-NOT
  \item[$CZ$] Porta controlada-Z

  \item[$M_m$] Operador de medição associado ao resultado $m$
  \item[$p(m)$] Probabilidade do resultado $m$ na medição

  \item[$G=(V,E)$] Grafo com vértices $V$ e arestas $E$
  \item[$C(z)$] Função custo do problema de otimização

  \item[$U(C,\gamma)$] Operador unitário associado ao hamiltoniano de custo
  \item[$U(B,\beta)$] Operador unitário associado ao hamiltoniano misturador
  \item[$F_p(\boldsymbol{\gamma},\boldsymbol{\beta})$] Valor esperado da função custo no nível $p$ do QAOA

  \item[$O(\cdot)$] Notação assintótica de complexidade

\end{simbolos}
% ---

% ---
% inserir o sumario
% ---
\pdfbookmark[0]{\contentsname}{toc}
\tableofcontents*
\cleardoublepage
% ---