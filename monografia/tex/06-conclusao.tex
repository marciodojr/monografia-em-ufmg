% !TEX root = Monografia.tex

\chapter[Conclusão]{Conclusão}

Neste trabalho foi proposto um estudo dos conceitos utilizados na computação quântica a partir da base matemática fornecida no curso de Especialização em Matemática ofertado pela UFMG. A computação quântica possui conexões bem amplas com a física, matemática e computação, sendo uma área desafiadora (e ao mesmo tempo intrigante) devido a maior profundidade dos conceitos utilizados e de certa forma, devido a "estranheza" criada pelas propriedades quânticas como superposição, emaranhamento entre outras.

Neste trabalho, primeiramente abordamos a computação quântica adequando o conhecimento matemático de espaços vetoriais, operadores lineares, produtos tensoriais às notações e conceitos da computação quântica, para esta parte as disciplinas do curso de Especialização em Matemática da UFMG foram essenciais, fornecendo uma base bastante sólida para o tema. O livro Quantum Computation and Quantum Information de Nielsen \& Chuang \cite{nielsen2010quantum} foi fundamental durante todo o estudo sendo um grande guia para os tópicos e para entendimento das notações utilizadas.

Em seguida, foi feita a apresentação dos postulados da mecânica quântica, os quais em conjunto com os conceitos matemáticos anteriores são fundamentais para o entendimento e manipulação de sistemas quânticos ou mais diretamente circuitos quânticos. Após a revisão dos postulados comparamos os conhecimentos sobre circuitos clássicos com o que a literatura e as empresas tem feito para implementar versões quânticas de circuitos, vimos que toda porta lógica quântica precisa ser reversivel e que empresas como IBM possuem bibliotecas (tal como o Qiskit) para construção, visualização, otimização, simulação e execução (em processadores quânticos reais) de circuitos quânticos. Tais bibliotecas aproveitam de muitas ferramentas clássicas e linguagens de programação existentes (tal como Python) para construir circuitos e simular ou se conectar com sistemas que executarão de fato o circuito em um processador quântico.

Por fim, fizemos um estudo de dois algoritmos quânticos. O objetivo de estudá-los foi o de aplicar o conhecimento na construção de alguns algoritmos bastante conhecidos, tentando, na medida do possível, justificar matematicamente seu comportamento. O QAOA em especial, apesar de ter uma implementação simples em Qiskit, parte de conceitos bastante avançados a respeito de simulação de sistemas quânticos em tempo contínuo e adequação para execução em número fixo de passos. Para este último algoritmo foi dada uma justificativa mais informal de seu comportamento.
