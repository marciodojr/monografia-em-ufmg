% !TEX root = Monografia.tex

\chapter[Os Postulados da Mecânica Quântica]{Os Postulados da Mecânica Quântica}
\label{chap:postulados}

\section{Introdução}

Este capítulo é baseado no conteúdo de \cite[seção ~2.2]{nielsen2010quantum}. O objetivo é apresentar os postulados e permitir ao leitor associar os conceitos iniciais de álgebra linear apresentados no capítulo 1 com os conceitos de mecânica quântica.

\section{Postulado 1 - Espaço de Estados}

A um sistema físico isolado associa-se um espaço de Hilbert complexo $\mathcal{H}$, chamado
\emph{espaço de estados} do sistema. O estado do sistema é completamente descrito por um vetor unitário
$$
\ket{\psi} \in \mathcal{H},
\qquad \langle \psi | \psi \rangle = 1.
$$
Vetores que diferem apenas por uma fase global $e^{i\theta}$ representam o mesmo estado físico,
isto é,
\begin{equation}\label{eq:gf}
\ket{\psi} \sim e^{i\theta}\ket{\psi}, \qquad \theta \in \mathbb{R}
\end{equation}

O sistema quântico mais simples é o \emph{qubit}. Um qubit possui um espaço de estados bidimensional. Seja $B=\{\ket{0}, \ket{1}\}$ base ortonormal de $\C^2$. Então, um vetor de estado arbitrário nesse espaço pode ser escrito como
\begin{equation}
\ket{\psi} = a \ket{0} + b \ket{1},
\end{equation}
onde $a$ e $b$ são números complexos tal que $|a|^2+|b|^2=1$. Os estados $\ket{0}$ e $\ket{1}$ podem ser vistos como os valores $1$ e $0$ que um bit pode assumir na computação clássica. No entanto, o qubit pode assumir superposições dos estados $\ket{0}$ e $\ket{1}$, ou seja, enquanto o bit pode assumir somente dois estados, um qubit pode assumir uma quantidade não enumerável de estados.

Podemos representar visualmente o qubit usando a \emph{Esfera de Bloch}. Qualquer estado de um qubit pode ser representado, a menos de uma fase global,
por um ponto na superfície da esfera. Por exemplo, Seja um estado arbitrário 
\begin{equation}
\ket{\psi} = a\ket{0} + b\ket{1},
\qquad a,b \in \mathbb{C},
\qquad |a|^2 + |b|^2 = 1.
\end{equation}
Escrevemos os coeficientes como
\begin{equation}
a = |a| e^{i\alpha}, \qquad b = |b| e^{i\beta},
\end{equation}
assim,
\begin{equation}
\ket{\psi}
= e^{i\alpha}\left(|a|\ket{0} + e^{i(\beta-\alpha)}|b|\ket{1}\right).
\end{equation}
Descartando a fase global $e^{i\alpha}$, que pode ser feito pela equação \eqref{eq:gf}, e definindo
$$
\phi = \beta - \alpha,
$$
obtemos um estado equivalente
\begin{equation}
\ket{\psi}
\sim |a|\ket{0} + e^{i\phi}|b|\ket{1}.
\end{equation}
Como $0 \le |a| \le 1$, existe um ângulo $\theta \in [0,\pi]$ tal que
\begin{equation}
|a| = \cos\!\left(\frac{\theta}{2}\right).
\end{equation}
Como $||\ket{\psi}||=1$, segue que
\begin{equation}
|b| = \sqrt{1-|a|^2} = \sen\!\left(\frac{\theta}{2}\right).
\end{equation}
Considerando que $e^{i\phi}=\cos \phi + i\sen \phi$, temos que
\begin{align}
\ket{\psi} &= \cos\!\left(\frac{\theta}{2}\right)\ket{0} + e^{i\phi}\sen\!\left(\frac{\theta}{2}\right)\ket{1} \\
           &= \cos\!\left(\frac{\theta}{2}\right)\ket{0} + \cos \phi \sen\!\left(\frac{\theta}{2}\right)\ket{1} + \sen \phi \sen\!\left(\frac{\theta}{2}\right)i\ket{1}.
\end{align}
Como o $\R$-espaço $\C^2$ é gerado pela base $\{\ket{0}, i\ket{0}, \ket{1}, i\ket{1}\}$ e $\ket{\psi}$ encontra-se no subespaço $3$-dimensional $\R^3$ gerado por $\{\ket{0}, \ket{1}, i\ket{1}\}$, representamos $\ket{\psi}$, para $0 \le \theta \le \pi$ e $0 \le \phi < 2\pi$, como o vetor unitário
$$
\vec r = (\sen\theta\cos\phi,\; \sen\theta\sen\phi,\; \cos\theta), r\in \R^3.
$$
Observe que ao definir as coordenadas de $\vec r$ dobramos o ângulo $\theta$, os vetores $\ket{0}$ e $\ket{1}$ são perpendiculares, e ao desenhar a esfera de Bloch fazemos uma representação dobrando o ângulo.

\begin{figure}[ht!]
    \centering
    \includegraphics[width=0.5\textwidth]{../../img/blochsphere.png}
    \caption{Estado $\ket{\psi}$ representado na Esfera de Bloch.}
    \label{fig:qubiteb}
\end{figure}

\FloatBarrier

\section{Postulado 2 - Evolução}\label{sec:post2}

A evolução temporal de um sistema quântico fechado é descrita por uma transformação unitária. Isto é, o estado do sistema no instante $t_1$, representado por
$\ket{\psi}$, está relacionado ao estado $\ket{\psi'}$ no instante $t_2$ por
\begin{equation}
\ket{\psi'} = U \ket{\psi},
\end{equation}
onde $U$ é um operador unitário que depende apenas do sistema e dos instantes $t_1$ e $t_2$.

\begin{example}[Operadores $X$ e $Z$]
Considere os operadores de Pauli
\begin{equation}
X =
\begin{bmatrix}
0 & 1 \\
1 & 0
\end{bmatrix} \qquad e \qquad
Z =
\begin{bmatrix}
1 & 0 \\
0 & -1
\end{bmatrix}.
\end{equation}
Estes operadores são unitários pois $XX^\dagger =X^\dagger X = I$ e $ZZ^\dagger = Z^\dagger Z$. O operador $X$, pode ser interpretado como \emph{operador NOT quântico}, invertendo os estados $\ket{0}$ e $\ket{1}$
\begin{equation}
X\ket{0} = \ket{1},
\qquad
X\ket{1} = \ket{0}.
\end{equation}
O operador $Z$ atua como um \emph{operador de mudança de fase}, deixando o estado $\ket{0}$ inalterado e introduzindo uma fase relativa ao ângulo $\pi$ no estado $\ket{1}$:
\begin{equation}
Z\ket{0} = \ket{0},
\qquad
Z\ket{1} = -\ket{1}.
\end{equation}
\end{example}

\begin{example}[Operador de Hadamard]\label{ex:h}
    O operador de Hadamard dado por
\begin{equation}
    H = \frac{1}{\sqrt{2}}\begin{bmatrix}
            1 & 1 \\
            1 & -1
        \end{bmatrix}
\end{equation}
é um exemplo de operador unitário. Este operador em especial é bastante utilizado em algoritmos quânticos para preparação de estados. Considere o estado
$$
\ket{\psi} = \ket{0},
$$
Então,
$$
H\ket{\psi} = H\ket{0} = \frac{1}{\sqrt{2}}(\ket{0} + \ket{1}) = \ket{+},
$$
onde $\ket{+}$ é o vetor definido no exemplo \ref{defket0}.
\end{example}

O Postulado 2 afirma que a evolução temporal de um sistema quântico fechado é descrita por uma transformação unitária. Para evoluções contínuas no tempo, essa transformação unitária é gerada pelo hamiltoniano do sistema. O hamiltoniano $H$ independente do tempo de um sistema quântico é um operador hermitiano que representa a energia total do sistema, de forma que a evolução temporal entre os instantes $0$ e $t$ é dada por
\begin{equation}
U(t) = e^{-iHt/\hbar}.
\end{equation}
Na última igualdade, $\hbar$ é a constante de Planck que, em geral, pode ser incorporada em $H$ de modo que $\tilde{H}=H/\hbar$ e
\begin{equation}
U(t) = e^{-i\tilde{H}t}.
\end{equation}
A unitariedade de $U(t)$ decorre diretamente do fato de que $H$ é hermitiano, pois
\begin{equation}
U(t)^\dagger U(t)
=
e^{i\tilde{H}t} e^{-i\tilde{H}t}
= I.
\end{equation}

\section{Postulado 3 - Medição Quântica}\label{sec:p3}

A medição quântica é descrita por um conjunto de operadores de medição $\{M_i\}_{i=1}^m$ que atuam no espaço de estados do sistema, satisfazendo a relação de
completude
\begin{equation}
\sum_{i=1}^m M_i^\dagger M_i = I.
\end{equation}
Observe que, não é necessário que o espaço de estados tenha dimensão igual ao número de operadores de medição (caso onde $\{M_i\}_{i=1}^m$ seriam projeções para autoespaços de dimensão igual a um), basta que a relação de completude seja satisfeita. Se o sistema está no estado $\ket{\psi}$ imediatamente antes da medição, então a probabilidade de se obter o resultado $i \in \{1, \dotso, m\}$ é dada por
\begin{equation}
p(i) = \braket{\psi | M_i^\dagger M_i | \psi}.
\end{equation}
Dado que o resultado $i$ é obtido, o estado do sistema após a medição passa a ser
\begin{equation}\label{eq:med}
\ket{\psi'} = \frac{M_i \ket{\psi}}{\sqrt{p(i)}}.
\end{equation}

\begin{example}[Medição do qubit na base computacional]
    A medição de um qubit na \emph{base computacional}, trata-se de uma medição em um único qubit, com dois resultados possíveis, definida pelos operadores de medição
\begin{equation}
M_0 = \ket{0}\bra{0} \qquad e \qquad M_1 = \ket{1}\bra{1}.
\end{equation}
Observe que cada operador de medição é hermitiano e satisfaz
\[
M_0^2 = M_0 \qquad e \qquad M_1^2 = M_1.
\]
Assim, a relação de completude é satisfeita:
\begin{equation}
I = M_0^\dagger M_0 + M_1^\dagger M_1 = M_0 + M_1.
\end{equation}
Suponha que o estado a ser medido seja
\begin{equation}
\ket{\psi} = a\ket{0} + b\ket{1}.
\end{equation}
Então, a probabilidade de se obter o resultado de medição $0$ é
\begin{equation}
p(0) = \braket{\psi | M_0^\dagger M_0 | \psi }
     = \braket{\psi | M_0 | \psi }
     = |a|^2.
\end{equation}
De modo análogo, a probabilidade de se obter o resultado de medição $1$ é
\begin{equation}
p(1) = |b|^2.
\end{equation}
O estado do sistema após a medição, nos dois casos, é portanto dado por
\begin{align}
\frac{M_0\ket{\psi}}{\sqrt{p(0)}} &= \frac{a}{|a|}\ket{0}, \\
\frac{M_1\ket{\psi}}{\sqrt{p(1)}} &= \frac{b}{|b|}\ket{1}.
\end{align}
Ou seja, o estado $\ket{\psi}$ ``colapsa'' para um estado equivalente a $\ket{0}$ ou $\ket{1}$ após a medição.

\end{example}

Um caso especial de operadores de medida são as projeções, vejamos o exemplo a seguir.

\begin{example}\label{ex:proj}
    Seja $M$ um operador hermitiano em um espaço de estados de um sistema sendo observado. Assuma que $M$ tem decomposição espectral
\begin{equation}
    M = \sum_{i=1}^n i P_i,
\end{equation}
de modo que $P_i$ é projeção sobre o autoespaço de $M$ associado ao autovalor $i$. Ao realizar a medição em $\ket{\psi}$, obtemos $i$ com probabilidade
\begin{equation}
    p(i) = \braket{\psi|P_i|\psi}.
\end{equation}
Dado que o resultado $i$ ocorre, o estado $\ket{\psi'}$ do sistema imediatamente após a medição é dado por
\begin{equation}
    \ket{\psi'} = \frac{P_i\ket{\psi}}{\sqrt{p(i)}}.
\end{equation}
\end{example}

Medições projetivas são bastante úteis pois simplificam alguns cálculos, por exemplo, o valor esperado (ou média) de um operador pode ser facilmente calculado. Por definição, a média de uma medida é dada por
\begin{equation}\label{eq:expec}
    E(M) = \sum_{i=1}^n i p(i) = \sum_{i=1}^n i \braket{\psi|P_i|\psi} = \braket{\psi|\sum_{i=1}^n i P_i|\psi} = \braket{\psi|M|\psi}.
\end{equation}
O desvio padrão $\Delta(M)$ para o operador $M$ é dado por
\begin{equation}
(\Delta M)^2
= \bra{\psi} M^2 \ket{\psi} - \braket{\psi | M |\psi}^2.
\end{equation}

\section{Postulado 4 - Sistemas Compostos}

O espaço de estados de um sistema quântico composto é o produto tensorial dos
espaços de estados dos sistemas componentes. Em particular, se dois sistemas
físicos têm espaços de estados $\mathcal{H}_A$ e $\mathcal{H}_B$, então o espaço de
estados do sistema composto é
\begin{equation}
\mathcal{H}_{AB} = \mathcal{H}_A \otimes \mathcal{H}_B.
\end{equation}

Além disso, se os sistemas $A$ e $B$ estão em estados $\ket{\psi_A}$ e $\ket{\psi_B}$,
respectivamente, então o estado do sistema composto é
\begin{equation}
\ket{\psi_{AB}} = \ket{\psi_A} \otimes \ket{\psi_B}.
\end{equation}

É importante ressaltar que nem todo estado em $\mathcal{H}_A \otimes \mathcal{H}_B$ pode ser escrito como um
produto tensorial de estados dos subsistemas. Estados que não admitem essa forma
são chamados de \emph{estados emaranhados} (\emph{entangled states}).

\begin{example}
Considere o estado de dois qubits
\begin{equation}
\ket{\beta_{00}}=\frac{1}{\sqrt{2}}\bigl(\ket{00} + \ket{11}\bigr)
\in \mathcal{H}_A \otimes \mathcal{H}_B.
\end{equation}
Suponha, por absurdo, que $\ket{\beta_{00}}$ pode ser escrito como um produto tensorial

\begin{equation}
    \ket{\beta_{00}} = \ket{\psi_A} \otimes \ket{\psi_B}, \qquad \ket{\psi_A} \in \mathcal{H}_A, \text{ } \ket{\psi_B} \in \mathcal{H}_B
\end{equation}
onde
\begin{equation}
    \ket{\psi_A} = a\ket{0} + b\ket{1} \qquad e \qquad \ket{\psi_B} = c\ket{0} + d\ket{1},
\end{equation}
com $a,b,c,d \in \C$. Assim,
\begin{align}
    \ket{\psi_A} \otimes \ket{\psi_B}
    &= (ac)\ket{00} + (ad)\ket{01} + (bc)\ket{10} + (bd)\ket{11}.
\end{align}
Temos então
\begin{equation}
    ac = \frac{1}{\sqrt{2}}, \qquad bd = \frac{1}{\sqrt{2}}, \qquad ad = 0, \qquad bc = 0.    
\end{equation}
Ou seja,
\begin{itemize}
    \item ou $a=0$ ou $d=0$,
    \item ou $b=0$ ou $c=0$.
\end{itemize}
O que contradiz $ac=bd=\frac{1}{\sqrt{2}}\neq 0$. Portanto, $\ket{\beta_{00}}$ não pode ser escrito como um produto tensorial de estados dos subsistemas $A$ e $B$.
\end{example}

O exemplo a seguir ilustra o uso de operadores em sistemas compostos.

\begin{example}
Considere um sistema composto por dois qubits, $A$ e $B$, cujos espaços de estados
são $\mathcal{H}_A = \mathbb{C}^2$ e $\mathcal{H}_B = \mathbb{C}^2$. O espaço
de estados do sistema composto é dado por
\begin{equation}
    \mathcal{H}_{AB} = \mathcal{H}_A \otimes \mathcal{H}_B = \C^2\otimes \C^2.
\end{equation}
Podemos tomar a base computacional deste sistema composto $B=\{\ket{00}, \ket{01}, \ket{10}, \ket{11}\}$ e considere o estado produto
\begin{equation}
\ket{\psi_{AB}} = \ket{0}_A \otimes \ket{1}_B = \ket{01}.
\end{equation}
Considere agora a aplicação do operador de Pauli $X$ sobre o qubit $A$, e o operador de Hadamard sobre o qubit $B$. O operador total que atua no sistema composto é
\begin{equation}
X \otimes H.
\end{equation}
Aplicando esse operador ao estado do sistema, obtemos
\begin{align}
(X \otimes H)\ket{01}
&= (X\ket{0}) \otimes (H\ket{1}) \\
&= \ket{1} \otimes \frac{1}{\sqrt{2}}(\ket{0} - \ket{1}) \\
&= \frac{1}{\sqrt{2}}(\ket{10}-\ket{11}).
\end{align}
\end{example}

Para sistemas compostos maiores é bastante comum indicar numericamente os sistemas componentes, assim, para um sistema composto de $n$ subsistemas a aplicação do operador $X$ na $i$-ésima componente é indicada por $X_i$.