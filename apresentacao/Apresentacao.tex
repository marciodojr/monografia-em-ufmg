\documentclass[aspectratio=169,11pt]{beamer}

% --- Idioma e fonte ---
\usepackage[T1]{fontenc}
\usepackage[utf8]{inputenc}
\usepackage[brazil]{babel}

% --- Matemática e símbolos ---
\usepackage{amsmath, amsfonts, amssymb}

% --- Figuras e layout ---
\usepackage{graphicx}
\usepackage{booktabs}

% --- Código ---
\usepackage{listings}
\usepackage{xcolor}

\lstset{
  basicstyle=\ttfamily\small,
  keywordstyle=\color{blue!70!black},
  commentstyle=\color{green!40!black},
  stringstyle=\color{orange!70!black},
  showstringspaces=false,
  breaklines=true,
  frame=single,
  tabsize=2
}

% --- Macros simples (evita dependências externas) ---
\newcommand{\C}{\mathbb{C}}
\newcommand{\R}{\mathbb{R}}
\newcommand{\ket}[1]{\lvert #1 \rangle}
\newcommand{\bra}[1]{\langle #1 \rvert}
\newcommand{\braket}[2]{\langle #1 | #2 \rangle}
\newcommand{\proj}[1]{\ket{#1}\!\bra{#1}}
\newtheorem{teorema}{Teorema}

\DeclareMathOperator{\sen}{sen}


% Idioma
\usepackage[brazil]{babel}
\usepackage[utf8]{inputenc}
\usepackage[T1]{fontenc}

% --- Tema ---
\usetheme{CambridgeUS}
\usecolortheme{dolphin}
\setbeamertemplate{navigation symbols}{}
\setbeamertemplate{caption}[numbered]


\usefonttheme{professionalfonts}

% --- Metadados (ajuste se quiser) ---
\title[Computação Quântica]{Um Estudo sobre Computação Quântica}
\author[Márcio Dias de Oliveira Junior]{Márcio Dias de Oliveira Junior}
\institute[UFMG -- ICEx]{
  Universidade Federal de Minas Gerais (UFMG)\\
  Instituto de Ciências Exatas (ICEx)\\
  Especialização em Matemática\\[2mm]
  Orientador: Prof. Dr. Csaba Schneider
}
\date[Belo Horizonte -- MG, 2026]{Belo Horizonte -- MG\\2026}

\begin{document}

% ------------------------------------------------------------
\begin{frame}
  \titlepage
\end{frame}

% ------------------------------------------------------------
\begin{frame}{Roteiro}
  \tableofcontents
\end{frame}

\section{Notação}

\begin{frame}{Notação Bra-Ket}

A notação $\ket{\psi}$ (ket) foi introduzida por Paul Dirac.

\medskip
\textbf{Interpretação:}
\begin{itemize}
    \item $\ket{\psi}$ representa um vetor em um espaço de Hilbert $\mathcal{H}$.
    \item $\bra{\psi}$ representa o funcional linear associado (vetor dual) relativo ao produto interno.
\end{itemize}

\medskip
Produto interno:
$$
\braket{\varphi}{\psi}
$$

\medskip
Assumindo um base ortonormal de $\mathcal{H}$ e com produto interno canônico podemos identificar:
$$
  \ket{\psi}=\begin{bmatrix}
    \psi_0 \\
    \psi_1 \\
    \vdots
  \end{bmatrix} \quad \text{e} \quad \bra{\psi} = [\psi_0^*, \psi_1^*, \dots].
$$
\end{frame}

%------------------------------------------------
% \begin{frame}{A base computacional}

% No caso de um qubit:

% $$
% \mathcal{H} = \mathbb{C}^2.
% $$

% A base computacional é definida por:

% $$
% \ket{0} =
% \begin{bmatrix}
% 1 \\
% 0
% \end{bmatrix},
% \qquad
% \ket{1} =
% \begin{bmatrix}
% 0 \\
% 1
% \end{bmatrix}.
% $$

% \medskip
% Propriedades:

% $$
% \braket{0}{0} = \braket{1}{1} = 1,
% \quad
% \braket{0}{1} = \braket{1}{0} = 0.
% $$

% \medskip
% Ou seja, $\{\ket{0},\ket{1}\}$ é uma base ortonormal.
% \end{frame}

% \begin{frame}{Espaços de Hilbert}
% Um \textbf{espaço de Hilbert} $\mathcal{H}$ é um espaço vetorial com produto interno $\braket{\cdot}{\cdot}$ e que é completo na norma induzida:
% $$
% \|\ket{\psi}\|=\sqrt{\braket{\psi}{\psi}}.
% $$

% \medskip
% \textbf{Exemplo (qubit):}\quad $\mathcal{H}=\mathbb{C}^2$ com base $\{\ket{0},\ket{1}\}$ e produto interno canônico.

% $$
% \ket{\psi} = a\ket{0} + b\ket{1}, \qquad |a|^2+|b^2|=1.
% $$

% \end{frame}

%------------------------------------------------
% \begin{frame}{Operadores lineares}
% Um \textbf{operador linear} $A:\mathcal{H}\to\mathcal{H}$ satisfaz
% $$
% A(\alpha\ket{\psi}+\beta\ket{\varphi})=\alpha A\ket{\psi}+\beta A\ket{\varphi}.
% $$

% \medskip
% \textbf{Adjunto:}\quad Dada a representação matricial de $A$ temos
% $$
% A^\dagger = (A^T)^* = (A^*)^T
% $$

% \medskip
% \textbf{Classes importantes:}
% \begin{itemize}
%     \item \textbf{Hermitiano:} $A=A^\dagger$.
%     \item \textbf{Unitário:} $U^\dagger U=I$.
%     \item \textbf{Projeção:} $P^2=P$ e $P^\dagger=P$.
% \end{itemize}
% \end{frame}

%------------------------------------------------
\begin{frame}{Operadores em base ortonormal (matriz)}
Fixe bases ortonormais $\{\ket{i}\}_{i=1}^{n}$ e $\{\ket{j}\}_{j=1}^{n}$.

\medskip
\textbf{Componentes (elementos de matriz):}
$$
A_{ij}=\braket{i}{A|j}.
$$

\medskip
\textbf{Expansão do operador (forma \textit{ket por bra}):}
$$
A=\sum_{i,j}^n \braket{i}{A |j}\ \ket{i}\bra{j} = \sum_{i,j}^n A_{ij} \ket{i}\bra{j}.
$$

\medskip
\textbf{Relação de completude para vetores ortonormais:}
$$
\sum_{i=0}^{n}\ket{i}\bra{i}=I
$$
\end{frame}

%------------------------------------------------
\begin{frame}{Produto tensorial}
Dados dois espaços de Hilbert $\mathcal{H}_A$ e $\mathcal{H}_B$ podemos construir o espaço tensorial (que também é um espaço de Hilbert):
$$
\mathcal{H}_{AB}=\mathcal{H}_A\otimes\mathcal{H}_B.
$$

% \medskip
% Se $\{\ket{i}\}_{i=1}^n$ é base de $\mathcal{H}_A$ e $\{\ket{j}\}_{j=1}^m$ é base de $\mathcal{H}_B$, então
% $\{\ket{i}\otimes\ket{j}\}_{i,i=1}^$ é base de $\mathcal{H}_A\otimes\mathcal{H}_B$.

% \medskip
% \textbf{Dimensão:}
% $$
% \dim(\mathcal{H}_A\otimes\mathcal{H}_B)=\dim(\mathcal{H}_A)\cdot\dim(\mathcal{H}_B) = n \cdot m.
% $$

\medskip
\textbf{Exemplo (2 qubits):}\quad $\mathbb{C}^2\otimes\mathbb{C}^2\simeq\mathbb{C}^4$ com base
$\{\ket{00},\ket{01},\ket{10},\ket{11}\}$, onde $\ket{00} = \ket{0}\otimes \ket{0}$, $\ket{01} = \ket{0}\otimes \ket{1} \dotso$

\end{frame}

%------------------------------------------------
% \begin{frame}{Operadores em espaços tensoriais}
% Se $A$ atua em $\mathcal{H}_A$ e $B$ atua em $\mathcal{H}_B$, define-se $A\otimes B$ por:
% $$
% (A\otimes B)(\ket{\psi}\otimes\ket{\varphi})=A\ket{\psi}\otimes B\ket{\varphi}.
% $$

% \medskip
% \textbf{Aplicação típica:} portas que atuam em um qubit de um registrador:
% $$
% U_k = I\otimes\cdots\otimes I\otimes U \otimes I\otimes\cdots\otimes I.
% $$

% \end{frame}

% \begin{frame}{Produto de Kronecker}

% O \textbf{produto de Kronecker} é a representação matricial do produto tensorial.

% \medskip
% Se $A \in \mathbb{C}^{m\times n}$ e $B \in \mathbb{C}^{p\times q}$, define-se

% $$
% A \otimes B =
% \begin{bmatrix}
% a_{11}B & a_{12}B & \cdots & a_{1n}B \\
% a_{21}B & a_{22}B & \cdots & a_{2n}B \\
% \vdots  & \vdots  & \ddots & \vdots \\
% a_{m1}B & a_{m2}B & \cdots & a_{mn}B
% \end{bmatrix}
% \in \mathbb{C}^{mp \times nq}.
% $$

% \medskip
% \textbf{Exemplo (2 qubits):}

% $$
% X \otimes I =
% \begin{bmatrix}
% 0 & 1 \\
% 1 & 0
% \end{bmatrix}
% \otimes
% \begin{bmatrix}
% 1 & 0 \\
% 0 & 1
% \end{bmatrix}
% =
% \begin{bmatrix}
% 0 & 0 & 1 & 0 \\
% 0 & 0 & 0 & 1 \\
% 1 & 0 & 0 & 0 \\
% 0 & 1 & 0 & 0
% \end{bmatrix}.
% $$

% \medskip
% \small
% Interpretação física: aplicar $X$ no primeiro qubit e identidade no segundo.
% \end{frame}

% ============================================================

\section{Postulados da mecânica quântica}

% ============================================================
\subsection{Postulado 1 - Espaço de estados}

\begin{frame}{Postulado 1 - Espaço de estados}
\textbf{Postulado 1.} A um sistema quântico isolado associa-se um \textbf{espaço de Hilbert complexo}
$\mathcal{H}$, chamado \emph{espaço de estados} do sistema. O estado do sistema é completamente descrito por um \textbf{vetor unitário} $\ket{\psi}\in\mathcal{H}$.

\medskip
\textbf{Equivalência (a menos de uma fase global).}
Estados que diferem por uma fase global representam o mesmo estado:
$$
\ket{\psi}\sim e^{i\alpha}\ket{\psi}.
$$

\end{frame}

% \begin{frame}{Postulado 1 - Qubit e parametrização (Esfera de Bloch)}
% Para um qubit, $\mathcal{H}=\mathbb{C}^2$, na base computacional $\{\ket{0},\ket{1}\}$:
% $$
% \ket{\psi}=a\ket{0}+b\ket{1},\qquad |a|^2+|b|^2=1.
% $$

% Removendo a fase global, pode-se escrever (a menos da equivalência):
% $$
% \ket{\psi}=
% \cos\!\left(\frac{\theta}{2}\right)\ket{0}
% +e^{i\varphi}\sin\!\left(\frac{\theta}{2}\right)\ket{1},
% \quad 0\le\theta\le\pi,\ 0\le\varphi<2\pi.
% $$

% \medskip
% \textbf{Leitura geométrica:} o par $(\theta,\varphi)$ identifica um ponto na Esfera de Bloch,
% com vetor
% $$
% \vec r=(\sin\theta\cos\varphi,\ \sin\theta\sin\varphi,\ \cos\theta).
% $$
% \end{frame}

% \begin{frame}{Postulado 1 - Qubit e parametrização (Esfera de Bloch)}
%   \begin{figure}[ht!]
%       \centering
%       \includegraphics[width=0.4\textwidth]{../img/blochsphere.png}
%       \caption{Estado $\ket{\psi}$ representado na Esfera de Bloch.}
%       \label{fig:aqubiteb}
%   \end{figure}
% \end{frame}

% ============================================================
\subsection{Postulado 2 - Evolução}

\begin{frame}{Postulado 2 - Evolução}
\textbf{Postulado 2.} A evolução temporal de um sistema quântico \textbf{fechado} é descrita por
uma \textbf{transformação unitária}. Se o estado no instante $t_1$ é $\ket{\psi}$, então no instante
$t_2$:
$$
\ket{\psi'} = U\ket{\psi},
$$
onde $U$ depende apenas do sistema e dos instantes $t_1,t_2$.

\medskip
\textbf{Consequências:}
\begin{itemize}
  \item Preservação de norma: $\|\ket{\psi'}\|=\|\ket{\psi}\|$.
  \item Reversibilidade: $U^{-1}=U^\dagger$.
  \item Portas quânticas em circuitos são operadores unitários.
\end{itemize}
\end{frame}

% \begin{frame}{Postulado 2 - Exemplos e evolução contínua}
% \textbf{Exemplos (Pauli).}
% $$
% X=\begin{bmatrix}0&1\\1&0\end{bmatrix},\quad
% Z=\begin{bmatrix}1&0\\0&-1\end{bmatrix}
% \quad\Rightarrow\quad
% X\ket{0}=\ket{1},\ X\ket{1}=\ket{0},\ Z\ket{1}=-\ket{1}.
% $$

% \medskip
% \textbf{Exemplo (Hadamard).}
% $$
% H=\frac{1}{\sqrt2}\begin{bmatrix}1&1\\1&-1\end{bmatrix},
% \qquad
% H\ket{0}=\frac{1}{\sqrt2}(\ket{0}+\ket{1})=\ket{+}.
% $$
% \end{frame}

% \begin{frame}{Postulado 2 - Evolução Contínua}

% A dinâmica contínua é governada pela \textbf{equação de Schrödinger dependente do tempo}:

% $$
% i\hbar \frac{d}{dt}\ket{\psi(t)} = H(t)\ket{\psi(t)},
% $$

% onde: $H(t)$ é o \textbf{Hamiltoniano} (operador hermitiano) e $\hbar$ é a constante de Planck.

% \medskip
% \textbf{Caso independente do tempo:}

% $$
% U(t_2,t_1) =
% \exp\!\left(
% -\frac{i}{\hbar} H (t_2 - t_1)
% \right).
% $$

% \medskip
% \textbf{Caso dependente do tempo:}

% $$
% U(t_2,t_1)
% =
% \mathcal{T}
% \exp\!\left(
% -\frac{i}{\hbar}
% \int_{t_1}^{t_2} H(s)\,ds
% \right),
% $$

% onde $\mathcal{T}$ é o operador de ordenação temporal.
% \end{frame}

% ============================================================
\subsection{Postulado 3 - Medição quântica}

\begin{frame}{Postulado 3 - Medição}
\textbf{Postulado 3.} Medições quânticas são descritas por um conjunto de operadores
$\{M_i\}_{i=1}^m$ (operadores de medição) que satisfazem a \textbf{relação de completude}:
$$
\sum_{i=1}^m M_i^\dagger M_i = I.
$$

Se o sistema está no estado $\ket{\psi}$:
\begin{itemize}
  \item \textbf{Probabilidade} do resultado $i$:
  $$
  p(i)=\bra{\psi}M_i^\dagger M_i\ket{\psi}.
  $$
  \item \textbf{Estado pós-medida} (condicionado ao resultado $i$):
  $$
  \ket{\psi'}=\frac{M_i\ket{\psi}}{\sqrt{p(i)}}.
  $$
\end{itemize}

\medskip
\textbf{Intuição:} A medição, em geral, introduz aleatoriedade.
\end{frame}

% \begin{frame}{Postulado 3 - Medição projetiva}
% \textbf{Medição projetiva.}
% Um caso importante ocorre quando $M_i=P_i$ são projetores ortogonais:
% $$
% P_i^2=P_i,\quad P_i^\dagger=P_i,\quad P_iP_j=0\ (i\neq j),\quad \sum_i P_i=I.
% $$

% \medskip
% \textbf{Exemplo: medição na base computacional.}
% $$
% M_0=\ket{0}\bra{0},\qquad M_1=\ket{1}\bra{1},\qquad I=M_0+M_1.
% $$
% Se
% $$
% \ket{\psi}=a\ket{0}+b\ket{1},
% $$
% então:
% $$
% p(0)=|a|^2,\quad p(1)=|b|^2,
% \qquad
% \ket{\psi_0}=\ket{0},\ \ket{\psi_1}=\ket{1}.
% $$

% \medskip
% \textbf{Em outras palavras:} Ao medir ``colapsamos'' o estado para um autovetor compatível com o resultado obtido.
% \end{frame}

% ============================================================
\subsection{Postulado 4 - Sistemas compostos}

\begin{frame}{Postulado 4 - Sistemas compostos}
\textbf{Postulado 4.} O espaço de estados de um sistema composto por dois subsistemas $A$ e $B$
é o \textbf{produto tensorial} dos espaços individuais:
$$
\mathcal{H}_{AB}=\mathcal{H}_A\otimes\mathcal{H}_B.
$$

Se $\{\ket{j}\}_{j=1}^n$ é base de $\mathcal{H}_A$ e $\{\ket{k}\}_{k=1}^m$ é base de $\mathcal{H}_B$, então
$\{\ket{j}\otimes\ket{k}\}_{i=j=1}^{j=n,k=m}$ é base de $\mathcal{H}_{AB}$.

\medskip
\textbf{Dimensão:}
$$
\dim(\mathcal{H}_A\otimes\mathcal{H}_B)=\dim(\mathcal{H}_A)\cdot \dim(\mathcal{H}_B).
$$

\medskip
\textbf{Operadores em sistemas compostos:}
$$
(A\otimes B)(\ket{\psi}\otimes\ket{\varphi})=A\ket{\psi}\otimes B\ket{\varphi}.
$$
\end{frame}

% \begin{frame}{Postulado 4 - Estados separáveis e emaranhamento}
% \textbf{Estados produto (separáveis).}
% Um estado é separável se pode ser escrito como
% $$
% \ket{\psi_{AB}}=\ket{\psi_A}\otimes\ket{\psi_B}.
% $$

% \medskip
% \textbf{Emaranhamento.}
% Existem estados do sistema composto que \emph{não} podem ser escritos como produto tensorial.
% Exemplo (estado de Bell):
% $$
% \ket{\beta_{00}}=\frac{1}{\sqrt{2}}\big(\ket{00}+\ket{11}\big)\in \mathbb{C}^2\otimes\mathbb{C}^2.
% $$

% \medskip
% \textbf{Ideia da prova (esboço):}
% assumindo $\ket{\beta_{00}}=(a\ket{0}+b\ket{1})\otimes(c\ket{0}+d\ket{1})$,
% os coeficientes exigiriam simultaneamente $ac=\frac{1}{\sqrt2}$, $bd=\frac{1}{\sqrt2}$,
% e $ad=bc=0$, o que é impossível.
% \end{frame}

% ============================================================

\section{Circuitos quânticos}

% ============================================================
\begin{frame}{Circuitos quânticos}

A computação quântica no modelo de circuitos é descrita por:

$$
\textbf{Estado inicial} \quad \longrightarrow \quad
\textbf{Portas unitárias} \quad \longrightarrow \quad
\textbf{Medição}
$$

\medskip
Formalmente:

$$
\ket{\psi_{\text{out}}}
=
U_k \cdots U_2 U_1 \ket{\psi_{\text{in}}}
$$

onde cada $U_i$ é unitário.

\medskip
\textbf{Componentes fundamentais:}
\begin{itemize}
    \item Registradores de qubits.
    \item Portas unitárias.
    \item Medição na base computacional.
\end{itemize}
\end{frame}

% ============================================================
% \begin{frame}{Registradores e estados iniciais}

% Um circuito com $n$ qubits atua sobre:

% $$
% \mathcal{H} = (\mathbb{C}^2)^{\otimes n}
% $$

% \medskip
% Estado inicial padrão:

% $$
% \ket{0}^{\otimes n}
% =
% \ket{00\cdots0}
% $$

% \medskip
% Estado geral:

% $$
% \ket{\psi}
% =
% \sum_{x\in\{0,1\}^n}
% \alpha_x \ket{x}
% \quad
% \text{com}
% \quad
% \sum_x |\alpha_x|^2 = 1.
% $$

% \medskip
% Dimensão do espaço:
% $$
% \dim(\mathcal{H}) = 2^n.
% $$
% \end{frame}

% ============================================================
% \begin{frame}{Portas de 1 qubit}

% Uma porta de 1 qubit é um operador unitário $U:\C^2\to \C^2$.

% \medskip
% Exemplos:

% $$
% X=
% \begin{bmatrix}
% 0 & 1 \\
% 1 & 0
% \end{bmatrix},
% \quad
% Z=
% \begin{bmatrix}
% 1 & 0 \\
% 0 & -1
% \end{bmatrix},
% $$

% $$
% H=\frac{1}{\sqrt{2}}
% \begin{bmatrix}
% 1 & 1 \\
% 1 & -1
% \end{bmatrix}.
% $$

% \medskip
% Aplicação em $n$ qubits:

% $$
% H_k =
% I \otimes \cdots \otimes H \otimes \cdots \otimes I.
% $$
% \end{frame}

% ============================================================
% \begin{frame}{Portas de múltiplos qubits}

% Portas que atuam simultaneamente em dois ou mais qubits
% introduzem correlações.

% \medskip
% Exemplo: \textbf{CNOT}

% $$
% \text{CNOT}\ket{x,y}
% =
% \ket{x, y \oplus x}.
% $$

% Matriz:

% $$
% \begin{bmatrix}
% 1&0&0&0\\
% 0&1&0&0\\
% 0&0&0&1\\
% 0&0&1&0
% \end{bmatrix}.
% $$

% \end{frame}

\begin{frame}{Emaranhamento: Estados de Bell}

\medskip
\textbf{Construção a partir de $\ket{00}$:}

Aplicando $H$ no primeiro qubit e depois $\text{CNOT}_{1,2}$:

\begin{align*}
  \ket{00} &\xrightarrow{H \otimes I} \frac{1}{\sqrt{2}}(\ket{00}+\ket{10})\\
           &\xrightarrow{\text{CNOT}} \frac{1}{\sqrt{2}}(\ket{00}+\ket{11}) \\
           &=\ket{\beta_{00}}.
\end{align*}

% \medskip
% \textbf{Os quatro estados de Bell:}

% \begin{align*}
% \ket{\beta_{00}}=\frac{1}{\sqrt{2}}(\ket{00}+\ket{11}), \quad \ket{\beta_{01}}=\frac{1}{\sqrt{2}}(\ket{01}+\ket{10}), \\
% \ket{\beta_{10}}=\frac{1}{\sqrt{2}}(\ket{00}-\ket{11}), \quad \ket{\beta_{11}}=\frac{1}{\sqrt{2}}(\ket{01}-\ket{10}).  
% \end{align*}

\end{frame}

% ============================================================
% \begin{frame}{Profundidade do circuito}

% \medskip
% \textbf{Profundidade:}
% número máximo de camadas de portas que devem ser aplicadas sequencialmente.

% O circuito a seguir tem profundidade 2.

%   \begin{figure}[ht!]
%       \centering
%       \includegraphics[width=0.4\textwidth]{../img/b00.png}
%       \caption{Circuito para geração do estado $\ket{\beta_{00}}$.}
%       \label{fig:ab00}
%   \end{figure}


% \end{frame}

% ============================================================
% \begin{frame}{Universalidade}

% Um conjunto de portas é \textbf{universal} se pode aproximar
% qualquer operador unitário com precisão arbitrária.

% \medskip
% Exemplo de conjunto universal:

% $$
% \{H, T, \text{CNOT}\}
% $$

% onde

% $$
% T=
% \begin{bmatrix}
% 1 & 0 \\
% 0 & e^{i\pi/4}
% \end{bmatrix}.
% $$

% \end{frame}

% ============================================================
\begin{frame}{Medição no modelo de circuitos}

Ao final do circuito, mede-se na base computacional:

$$
\{\ket{x}\bra{x}\}_{x\in\{0,1\}^n}.
$$

\medskip
Probabilidade de obter $x$:

$$
P(x) = |\alpha_x|^2.
$$

\medskip
Resultado clássico produzido pelo circuito:

$$
x \in \{0,1\}^n.
$$

\medskip
Um circuito quântico implementa uma \textbf{distribuição de probabilidade} sobre sequências binárias.
\end{frame}

% \begin{frame}{Teorema da não clonagem}
%   \begin{teorema}[Não clonagem para um qubit]\label{th:clone}
%     Não existem $\ket{s}\in \C^2$ e $U:\C^4\to \C^4$, um operador unitário, tais que $U(\ket{\psi}\otimes \ket{s}) = \ket{\psi}\otimes\ket{\psi}$ para todo $\ket{\psi}\in \C^2$. Em outras palavras, não é possível construir um circuito quântico que realize a cópia de um qubit arbitrário $\ket{\psi}$.
%   \end{teorema}
% \end{frame}

% \begin{frame}{Teorema da não clonagem}
%     Se a cópia funciona para os estados $\ket{\psi}$ e $\ket{\varphi}$, temos
%         %
%             \begin{align*}
%                 U(\ket{\psi}\otimes\ket{s}) = \ket{\psi}\otimes\ket{\psi} \\
%                 U(\ket{\varphi}\otimes\ket{s}) = \ket{\varphi}\otimes\ket{\varphi}.
%             \end{align*}
%         %
%         Ao tomar o produto interno das equações acima temos
%         %
%         \begin{align*}
%             \braket{U(\ket{\psi}\otimes\ket{s})}{U(\ket{\varphi}\otimes\ket{s})}  &= \braket{(\ket{\psi}\otimes\ket{s})}{(\ket{\varphi}\otimes\ket{s})} \\
%                         &=\braket{\psi}{\varphi} \\
%                         &= \braket{(\ket{\psi}\otimes\ket{\varphi})}{(\ket{\psi}\otimes\ket{\varphi})} \\
%                         &= \braket{\psi}{\varphi}^2,
%         \end{align*}
%         %
%         portanto,
%         %
%         \begin{equation*}
%             \braket{\psi}{\varphi}= \braket{\psi}{\varphi}^2 \implies \braket{\psi}{\varphi}=0 \text{ ou } \braket{\psi}{\varphi}=1,
%         \end{equation*}
%         que só é possível se $\ket{\varphi}=\ket{\psi}$ ou se $\ket{\varphi}$ e $\ket{\psi}$ são ortogonais.
% \end{frame}

\begin{frame}{Teleporte quântico}
\textbf{Objetivo:} transferir um qubit desconhecido
$$
\ket{\psi}=a\ket{0}+b\ket{1}
$$
de Alice para Bob sem um canal de comunicação quântica.

\medskip
\textbf{Cenário:}
\begin{itemize}
    \item Alice e Bob compartilham previamente um par EPR (assuma $\ket{\beta_{00}}$).
    \item Anos depois, Alice precisa entregar o estado $\ket{\psi}$ a Bob.
    \item Alice pode enviar apenas \textbf{informação clássica} para Bob.
\end{itemize}

\medskip
\textbf{Ideia:} usar emaranhamento + 2 bits clássicos + operações locais em Bob.
\end{frame}

%------------------------------------------------
% \begin{frame}{Pontos importantes}
% \textbf{Dificuldades fundamentais:}
% \begin{itemize}
%     \item Alice possui \emph{apenas uma cópia} de $\ket{\psi}$.
%     \item Pelo \textbf{teorema da não-clonagem}, ela não pode copiar $\ket{\psi}$.
%     \item Uma medição direta altera o estado, pois o estado pós-medida depende do resultado.
% \end{itemize}

% \medskip
% Mesmo que Alice conhecesse $\ket{\psi}$, descrevê-lo exatamente exigiria especificar parâmetros em um conjunto contínuo.

% \medskip
% \textbf{Solução:} explorar o par EPR compartilhado para reconstruir $\ket{\psi}$ em Bob.
% \end{frame}

%------------------------------------------------
\begin{frame}{Circuito do teleporte}
\begin{figure}
    \centering
    \includegraphics[width=0.7\textwidth]{../img/qtel.png}
    \caption{Circuito quântico para teleporte de um qubit.}
    \label{fig:aqtel_beamer}
\end{figure}

\medskip
\textbf{Resumo operacional:}
\begin{itemize}
    \item Alice aplica CNOT e Hadamard em seus qubits, mede e obtém dois bits.
    \item Alice envia os 2 bits a Bob (canal clássico).
    \item Bob aplica uma correção $I, X, Z$ ou $XZ$.
\end{itemize}
\end{frame}

%------------------------------------------------
% \begin{frame}{Estado de entrada e aplicação do CNOT}
% Assuma $\ket{\psi}=a\ket{0}+b\ket{1}$ e o par EPR $\ket{\beta_{00}}=\frac{1}{\sqrt2}(\ket{00}+\ket{11})$.

% \medskip
% \textbf{Estado de entrada:}
% \begin{align*}
% \ket{\psi_0}
% &=\ket{\psi}\otimes\ket{\beta_{00}}\\
% &=\frac{1}{\sqrt2}\Big(a\ket{0}\otimes(\ket{00}+\ket{11})+b\ket{1}\otimes(\ket{00}+\ket{11})\Big).
% \end{align*}

% \medskip
% Convenção: os dois primeiros qubits são de Alice e o terceiro é de Bob.

% \medskip
% \textbf{Após CNOT (controle = qubit $\ket{\psi}$; alvo = metade do EPR de Alice):}
% $$
% \ket{\psi_1}=\frac{1}{\sqrt2}\Big(a\ket{0}\otimes(\ket{00}+\ket{11})+b\ket{1}\otimes(\ket{10}+\ket{01})\Big).
% $$
% \end{frame}

%------------------------------------------------
% \begin{frame}{Aplicação do Hadamard e decomposição em 4 casos}
% Após aplicar Hadamard no primeiro qubit de Alice:
% \begin{equation*}
% \begin{split}
% \ket{\psi_2}
% &=\frac{1}{2}\Big(a(\ket{0}+\ket{1})\otimes(\ket{00}+\ket{11})
% +b(\ket{0}-\ket{1})\otimes(\ket{10}+\ket{01})\Big)\\
% &=\frac{1}{2}\Big( \\
% &\qquad +\ket{00}(a\ket{0}+b\ket{1})  \\
% &\qquad +\ket{01}(a\ket{1}+b\ket{0}) \\
% &\qquad +\ket{10}(a\ket{0}-b\ket{1}) \\
% &\qquad +\ket{11}(a\ket{1}-b\ket{0}) \\
% &\quad\Big).
% \end{split}
% \end{equation*}

% \medskip
% Os qubits de Alice determinam um dentre quatro termos (resultados clássicos $00,01,10,11$).
% \end{frame}

%------------------------------------------------
% \begin{frame}{Medição de Alice, comunicação clássica e correções de Bob}
% \textbf{Estado de Bob condicionado ao resultado de Alice:}
% \begin{align*}
% 00 &\mapsto \ket{\psi_3(00)} = a\ket{0}+b\ket{1} \\
% 01 &\mapsto \ket{\psi_3(01)} = a\ket{1}+b\ket{0} \\
% 10 &\mapsto \ket{\psi_3(10)} = a\ket{0}-b\ket{1} \\
% 11 &\mapsto \ket{\psi_3(11)} = a\ket{1}-b\ket{0}
% \end{align*}

% \medskip
% \textbf{Correção em Bob:}
% \begin{itemize}
%     \item $00$: $I$ (não faz nada)
%     \item $01$: aplica $X$
%     \item $10$: aplica $Z$
%     \item $11$: aplica $X$ e depois $Z$
% \end{itemize}

% \end{frame}

\begin{frame}{Teleporte quântico (Qiskit)}
  Implementação em Qiskit (\texttt{quantum-teleportation.ipynb}).

  \begin{figure}[ht!]
      \centering
      \includegraphics[width=0.95\textwidth]{../img/tqdraw.png}
      \caption{Diagrama do circuito para o teleporte quântico.}
      \label{fig:atqdraw}
  \end{figure}
\end{frame}

\begin{frame}{Teleporte quântico (Qiskit)}
  Qubit teleportado sempre aparece com resultado $\ket{0}$.

    \begin{figure}[ht!]
      \centering
      \includegraphics[width=0.5\textwidth]{../img/tqsimdraw.png}
      \caption{Histograma para simulação de medições no circuito do teleporte quântico.}
      \label{fig:atqsimdraw}
  \end{figure}
\end{frame}

\begin{frame}{Comparativo}
\begin{columns}[T]
  \column{0.48\textwidth}
  \textbf{Computação Clássica}
  \begin{itemize}
      \item Unidade básica: bit (0 ou 1)
      \item Estados discretos em $\{0,1\}^n$
      \item Funções booleanas
      \item Operações podem ser irreversíveis
      \item Cópia de bits
      \item Medição não altera o estado
  \end{itemize}

  \column{0.48\textwidth}
  \textbf{Computação Quântica}
  \begin{itemize}
      \item Unidade básica: qubit ($\ket{0}, \ket{1}$)
      \item Espaço de Hilbert de dimensão $2^n$
      \item Operadores unitários
      \item Todas as operações são reversíveis
      \item Não clonagem de qubits
      \item Medição colapsa o estado
  \end{itemize}
\end{columns}
\end{frame}

\section{Exemplos de algoritmos quânticos}

\subsection{Algoritmo de Grover}

\begin{frame}{Problema de Busca Não-Estruturada}

Seja $N=2^n$ e o conjunto de estados computacionais
$$
\{x \mid x \in \{0,1\}^n\}.
$$

Desejamos encontrar $x^\star$ tal que

$$
f(x^\star)=\begin{cases}
    1, \quad \text{ se } x = x^\star \\
    0 , \quad \text{ caso contrário}.
  \end{cases}
$$

\textbf{Oráculo quântico:}
$$
U_f\ket{x} = (-1)^{f(x)}\ket{x}.
$$

\begin{itemize}
  \item $U_f$ aplica fase $-1$ apenas no estado marcado.
  \item Não há estrutura conhecida em $f$.
  \item Busca clássica: $O(N)$.
\end{itemize}

\end{frame}

%------------------------------------------------
\begin{frame}{Estado inicial}

Aplicamos Hadamard em todos os qubits:

$$
\ket{s} = H^{\otimes n}\ket{0}^{\otimes n}
= \frac{1}{\sqrt{N}}\sum_{x\in \{0,1\}^n}\ket{x}.
$$

Definimos o estado ortogonal ao estado marcado:

$$
\ket{x^\star_\perp} = \frac{1}{\sqrt{N-1}}\sum_{x\neq x^\star}\ket{x}.
$$

O estado $\ket{s}$ pertence ao subespaço $V$ gerado por $\{\ket{x^\star}, \ket{x^\star_\perp}\}$:

$$
\ket{s} = \sen(\theta)\ket{x^\star} + \cos(\theta)\ket{x^\star_\perp}, \qquad \sen(\theta)=\frac{1}{\sqrt{N}}, \quad \cos(\theta)=\sqrt{\frac{N-1}{N}}.
$$

\end{frame}

% %------------------------------------------------
\begin{frame}{Ação do Oráculo no subespaço}

No subespaço $V$:

$$
U_f =
\begin{bmatrix}
-1 & 0 \\
0 & 1
\end{bmatrix}, \qquad \ket{s} = \begin{bmatrix}
  \sen \theta \\
  \cos \theta
\end{bmatrix}
$$

\begin{itemize}
\item Inverte o sinal do estado marcado.
\item Mantém os demais estados invariantes.
\item A dinâmica ocorre apenas em um espaço de dimensão 2.
\end{itemize}

\end{frame}

%------------------------------------------------
\begin{frame}{Operador difusor}

Definimos no subespaço $V$

$$
D = 2\ket{s}\!\bra{s} - I. = \begin{bmatrix}
  -\cos(2\theta) & \sen(2\theta) \\
  \sen(2\theta) & \cos(2\theta)
\end{bmatrix}.
$$

Como $\ket{s}\bra{s}=\big(H^{\otimes^n}\ket{0}\big)\big(\bra{0}H^{\otimes^n}\big)$, podemos implementar $D$ como:

$$
D =
H^{\otimes n}
(2\ket{0}\!\bra{0}-I)
H^{\otimes n}.
$$

\textbf{Interpretação:} reflexão em torno de $\ket{s}$.

\end{frame}

% %------------------------------------------------
\begin{frame}{Iteração de Grover}

Definimos:

$$
G = D U_f.
$$


No subespaço $V$:

$$
G =
\begin{bmatrix}
\cos(2\theta) & \sen(2\theta) \\
-\sen(2\theta) & \cos(2\theta)
\end{bmatrix}.
$$

\begin{itemize}
\item $G$ é uma rotação no plano $V$.
\item A cada iteração, a amplitude do estado marcado aumenta.
\end{itemize}

\end{frame}

%------------------------------------------------
\begin{frame}{Evolução Após $r$ iterações}

$$
G^r\ket{s}
=
\sen((2r+1)\theta)\ket{x^\star}
+
\cos((2r+1)\theta)\ket{x^\star_\perp}.
$$

Probabilidade de medir o estado marcado:

$$
P(r)=\sen^2((2r+1)\theta).
$$

\textbf{Objetivo:} maximizar $P(r)$.

\end{frame}

% %------------------------------------------------
\begin{frame}{Número ótimo de iterações}

Queremos:

$$
(2r+1)\theta \approx \frac{\pi}{2}.
$$

Logo,

$$
r \approx \frac{\pi}{4\theta}-\frac{1}{2}.
$$

Como $\theta \approx \frac{1}{\sqrt{N}}$:

$$
r \approx
\left\lfloor
\frac{\pi}{4}\sqrt{N}
\right\rfloor.
$$

O algoritmo encontra $x^\star$ com alta probabilidade usando $O(\sqrt{N})$ invocações do oráculo.

\end{frame}

%------------------------------------------------
% \begin{frame}{Caso com $M$ Soluções}

% Definimos

% $$
% \ket{x^\star} = \frac{1}{\sqrt{M}}\sum_{f(x)=1}\ket{x}, \qquad \ket{x^\star_\perp} = \frac{1}{\sqrt{N-M}}\sum_{f(x)=0}\ket{x}.
% $$

% Assim,

% $$
% \ket{s} = \frac{1}{\sqrt{N}} \sum_{x\in \{0, 1\}^n} \ket{x} = \frac{M}{N}\ket{x^\star} + \frac{N-M}{N}\ket{x^\star_\perp}.
% $$

% Tomando $\sen\theta = \sqrt{M/N}$ obtemos de forma análoga

% $$
% r \approx
% \left\lfloor
% \frac{\pi}{4}\sqrt{\frac{N}{M}}
% \right\rfloor.
% $$

% \end{frame}

\begin{frame}{Algoritmo de Grover (Qiskit)}
  Implementação em Qiskit (\texttt{grovers-algorithm.ipynb}).

  \begin{figure}[ht!]
      \centering
      \includegraphics[width=0.65\textwidth]{../img/alggrover.png}
      \caption{Diagrama do circuito quântico para o algoritmo de Grover para quatro qubits.}
      \label{fig:agrovercirc}
  \end{figure}
\end{frame}

\begin{frame}{Algoritmo de Grover (Qiskit)}
  Estados marcados aparecem com maior frequência.

  \begin{figure}[ht!]
      \centering
      \includegraphics[width=0.5\textwidth]{../img/groveralghist.png}
      \caption{Histograma de medição para o algoritmo de Grover.}
      \label{fig:agroveralghist}
  \end{figure}
\end{frame}

\subsection{QAOA}

%------------------------------------------------
\begin{frame}{Problemas de Otimização Combinatória}
Seja $X=\{0,1\}^n$ o conjunto de soluções $z=z_1z_2\cdots z_n \in \{0,1\}^n$ (sequências binárias). Um problema de otimização combinatória é dado por funções locais
$$
C_\alpha: X\to \{0,1\},\qquad \alpha=1,\dots,m,
$$
e pela função objetivo
$$
C(z)=\sum_{\alpha=1}^m C_\alpha(z).
$$

\begin{itemize}
  \item Objetivo: encontrar $z$ que maximize $C(z)$ (solução ótima).
  \item Otimização aproximada: buscar $z$ com $C(z)$ próximo do máximo.
  \item Hipótese de \emph{localidade}: cada $C_\alpha$ é implementável com $O(1)$ portas.
\end{itemize}
\end{frame}

%------------------------------------------------
\begin{frame}{Mapeamento para um espaço de Hilbert}
Considere o espaço de Hilbert gerado pelas sequências binárias de comprimento $n$
$$
\mathcal{H} = \{\ket{z}: z\in X\},\qquad \dim(\mathcal{H})=2^n,
$$
com decomposição
$$
\mathcal{H}=\mathcal{H}_1\otimes\cdots\otimes \mathcal{H}_n,
$$
onde cada $H_i$ tem dimensão $2$ (um qubit).

A função objetivo define um operador diagonal (hermitiano) na base computacional:
$$
C=\sum_{z\in X} C(z)\ket{z}\bra{z}.
$$

Logo, $\ket{z}$ são autovetores de $C$:
$$
C\ket{z}=C(z)\ket{z}.
$$
\end{frame}

%------------------------------------------------
\begin{frame}{Operador de custo}
Como $C$ é hermitiano, definimos o operador unitário de custo:
$$
U(C,\gamma)=\prod_{\alpha=1}^m e^{-i\gamma C_\alpha}=e^{-i\gamma C},
\qquad \gamma\in[0,2\pi).
$$

\begin{itemize}
\item Em geral, produtos de exponenciais podem não comutar.
\item Aqui, cada $C_\alpha$ é diagonal na base computacional $\Rightarrow$ comutam entre si.
\item Implementação em circuito: profundidade máxima $O(m)$ (uma camada por termo local).
\end{itemize}
\end{frame}

%------------------------------------------------
\begin{frame}{Operador de mistura}
Defina, para cada qubit $j$,
$$
X_j = I\otimes\cdots\otimes I \otimes X \otimes I\otimes\cdots\otimes I,
$$
e
$$
B=\sum_{j=1}^n X_j.
$$

O operador unitário de mistura é
$$
U(B,\beta)=\prod_{j=1}^n e^{-i\beta X_j}=e^{-i\beta B},
\qquad \beta\in[0,\pi).
$$

Além disso, podemos implementar $U(B, \beta)$ com $O(1)$ portas
$$
U(B,\beta)=\bigotimes_{j=1}^n R_x(2\beta).
$$
\end{frame}

%------------------------------------------------
\begin{frame}{Inicialização: estado uniforme}
Inicializamos em
$$
\ket{\psi}=\ket{0}^{\otimes n},
$$
e aplicamos $H^{\otimes n}$:
$$
\ket{s}=H^{\otimes n}\ket{0}^{\otimes n}
=\frac{1}{\sqrt{2^n}}\sum_{z\in X}\ket{z}.
$$

\end{frame}

%------------------------------------------------
\begin{frame}{Estado variacional do QAOA (profundidade $p$)}
Para $p\ge 1$, dados ângulos
$$
\gamma_1,\dots,\gamma_p,\qquad \beta_1,\dots,\beta_p,
$$
definimos o estado
$$
\ket{\gamma,\beta}
=
U(B,\beta_p)U(C,\gamma_p)\cdots U(B,\beta_1)U(C,\gamma_1)\ket{s}.
$$

Forma compacta:
$$
\ket{\gamma,\beta}=\prod_{l=p}^1 e^{-i\beta_l B}e^{-i\gamma_l C}\,\ket{s}.
$$

\begin{itemize}
\item Cada camada aplica (custo) $\rightarrow$ (mistura).
\item Profundidade máxima: $O(mp)$ (custo: $O(m)$; mistura: $O(1)$).
\end{itemize}
\end{frame}

%------------------------------------------------
\begin{frame}{Função objetivo do QAOA}
A qualidade do estado é medida pela expectativa ou média:
$$
F_p(\gamma,\beta)=\braket{\gamma,\beta}{C|\gamma,\beta}.
$$

Definimos o melhor valor atingível com profundidade $p$:
$$
M_p=\max_{\gamma,\beta} F_p(\gamma,\beta).
$$

Monotonicidade:
$$
M_p\ge M_{p-1},
$$
pois é sempre possível escolher $\gamma_p=\beta_p=0$ e “simular” a profundidade menor.
\end{frame}

%------------------------------------------------
\begin{frame}{Limite superior via princípio variacional}
Se $C$ tem decomposição espectral
$$
C=\sum_{i=1}^m c_i\ket{e_i}\bra{e_i},\qquad c_1\ge c_2\ge\cdots\ge c_m\ge 0,
$$
então para qualquer estado normalizado $\ket{\psi}$:
$$
\bra{\psi}C\ket{\psi}\le c_1,
\quad \text{e}\quad
\max_{\|\psi\|=1}\bra{\psi}C\ket{\psi}=c_1.
$$

\begin{itemize}
\item O valor ótimo é atingido no autoespaço do maior autovalor.
\item No QAOA, os estados permitidos são restritos à família $\ket{\gamma,\beta}$.
\end{itemize}
\end{frame}

%------------------------------------------------
\begin{frame}{Aumentando o número de passos $p$}
Em geral, pode não existir $(\gamma,\beta)$ tal que
$$
\ket{\gamma,\beta}=\ket{e_1}.
$$

No entanto, aumentando $p$, a família de estados variacionais enriquece e (sob hipóteses do artigo) obtém-se:
$$
\lim_{p\to\infty} M_p=\max_{z\in X} C(z).
$$

\begin{itemize}
\item Interpretação: maior expressividade do circuito com mais camadas.
\item A seguir: conexão informal com evolução adiabática + discretização.
\end{itemize}
\end{frame}

%------------------------------------------------
\begin{frame}{Fórmula de Lie--Trotter (ponte para discretização)}
Para operadores hermitianos $B$ e $C$,
$$
e^{\, i(B+C)\Delta t}=e^{\, iB\Delta t}\,e^{\, iC\Delta t}\;+\;O(\Delta t^2),
$$
onde o erro é entendido na norma de operadores:
$$
\|e^{i(B+C)\Delta t}-e^{iB\Delta t}e^{iC\Delta t}\|\le c\,\Delta t^2.
$$

\begin{itemize}
\item Aproxima a exponencial do hamiltoniano pelo produto de exponenciais.
\item Permite implementar evolução usando blocos simples (custo/mistura).
\end{itemize}
\end{frame}

%------------------------------------------------
\begin{frame}{Evolução adiabática: visão contínua (QAA)}
Com $\hbar=1$, a evolução temporal com hamiltoniano $H(t)$ é
$$
U(t_1,t_2)=\mathcal{T}\exp\!\left(-i\int_{t_1}^{t_2} H(t)\,dt\right),
\qquad
\ket{\psi(T)}=U(T)\ket{\psi(0)}.
$$

No QAA, usa-se interpolação linear:
$$
H(t)=\left(1-\frac{t}{T}\right)B+\frac{t}{T}C,
\quad t\in[0,T],
$$
com $H(0)=B$ e $H(T)=C$.

Teorema adiabático (ideia): para $T$ grande e boa lacuna espectral,
$$
\ket{\psi(T)} \approx \text{autovetor de maior autovalor de } C.
$$
\end{frame}

%------------------------------------------------
\begin{frame}{Discretização do QAA}
Divida $[0,T]$ em $p$ subintervalos de tamanho $\Delta t=T/p$ e pontos $t_l=l\Delta t$.

$$
U(T)=\prod_{l=p-1}^{0} U(t_l,t_{l+1}).
$$

Aproximando o hamiltoniano por seu valor em $t_l$:
$$
U(t_l,t_{l+1}) \approx e^{-i\Delta t\big((1-s_l)B+s_lC\big)},
\qquad s_l=\frac{l}{p}.
$$

Aplicando Lie--Trotter:
$$
e^{-i\Delta t\big((1-s_l)B+s_lC\big)}
\approx e^{-i\beta_l B}\,e^{-i\gamma_l C},
$$
com
$$
\beta_l=\Delta t(1-s_l),\qquad \gamma_l=\Delta t\,s_l.
$$
\end{frame}

%------------------------------------------------
\begin{frame}{Discretização do QAA}
Assim,
$$
  U(T)\ket{s}
  \approx
  \left(\prod_{l=p-1}^{0} e^{-i\beta_l B}e^{-i\gamma_l C}\right)\ket{s}
  =\ket{\gamma,\beta}.
$$

  \begin{itemize}
  \item QAOA $\approx$ discretização de uma evolução adiabática contínua.
  \item Parâmetros $(\gamma_l,\beta_l)$ codificam “passos” da interpolação.
  \item Para $p$ grande (passos pequenos), aproxima-se a evolução contínua.
  \end{itemize}
\end{frame}

%------------------------------------------------
\begin{frame}{Considerações}
  \begin{itemize}
  \item QAOA constrói estados variacionais com camadas alternadas:
  $$
  U(C,\gamma_l) \ \text{(custo)}\quad\text{e}\quad U(B,\beta_l)\ \text{(mistura)}.
  $$
  \item Otimizando os $2p$ ângulos maximizamos
  $$
  F_p(\gamma,\beta)=\braket{\gamma,\beta}{C|\gamma,\beta}.
  $$
  \item Aumentar $p$ aumenta expressividade e conecta-se ao limite adiabático:
  $$
  \lim_{p\to\infty} M_p=\max_{z\in X} C(z).
  $$
  \end{itemize}
\end{frame}

\begin{frame}{QAOA (Qiskit)}
  Grafo para busca do corte máximo

    \begin{figure}[ht!]
        \centering
        \includegraphics[width=0.40\textwidth]{../img/gqaoa.png}
        \caption{Grafo para uso no QAOA.}
        \label{fig:agqaoa}
    \end{figure}
\end{frame}

\begin{frame}{QAOA (Qiskit)}
  Implementação em Qiskit (\texttt{qaoa.ipynb}).

    \begin{figure}[ht!]
        \centering
        \includegraphics[width=0.9\textwidth]{../img/qaoaex.png}
        \caption{Circuito do QAOA.}
        \label{fig:aqaoaexd}
    \end{figure}
\end{frame}

\begin{frame}{QAOA (Qiskit)}
  Corte máximo aparece com maior frequência.

  \begin{figure}[ht!]
      \centering
      \includegraphics[width=0.5\textwidth]{../img/qaoafim.png}
      \caption{Histograma de medida com o resultado correto para o corte máximo.}
      \label{fig:aqaoafim}
  \end{figure}
\end{frame}

\begin{frame}{Agradecimentos}

\begin{center}
\small Muito obrigado!
\end{center}

\end{frame}

\end{document}
